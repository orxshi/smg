\documentclass{article}

\usepackage{tikz, tikz-3dplot}

\newcommand{\clen}{4.0}

\begin{document}

\begin{figure}
\centering

\tdplotsetmaincoords{60}{130}

\begin{tikzpicture}[scale=2,tdplot_main_coords]
\coordinate (O) at (0,0,0);
\tdplotsetcoord{P}{1.0}{45}{45}
\draw[thick,->] (0,0,0) -- (1,0,0) node[anchor=north east]{$x(i)$};
\draw[thick,->] (0,0,0) -- (0,1,0) node[anchor=north west]{$y(j)$};
\draw[thick,->] (0,0,0) -- (0,0,1) node[anchor=south]{$z(k)$};

\draw[dashed, color=red] (O) -- (Px);
\draw[dashed, color=red] (O) -- (Py);
\draw[dashed, color=red] (O) -- (Pz);
\draw[dashed, color=red] (Px) -- (Pxy);
\draw[dashed, color=red] (Py) -- (Pxy);
\draw[dashed, color=red] (Px) -- (Pxz);
\draw[dashed, color=red] (Pz) -- (Pxz);
\draw[dashed, color=red] (Py) -- (Pyz);
\draw[dashed, color=red] (Pz) -- (Pyz);
\draw[dashed, color=red] (Pxy) -- (P);
\draw[dashed, color=red] (Pxz) -- (P);
\draw[dashed, color=red] (Pyz) -- (P);
\end{tikzpicture}


\end{figure}

\section{Making a cell}

\begin{figure}
\centering

\tdplotsetmaincoords{60}{130}

\begin{tikzpicture}[scale=2,tdplot_main_coords]
\coordinate (O) at (0,0,0);
\tdplotsetcoord{P}{1.0}{45}{45}
%\draw[thick,->] (0,0,0) -- (1,0,0) node[anchor=north east]{$x(i)$};
%\draw[thick,->] (0,0,0) -- (0,1,0) node[anchor=north west]{$y(j)$};
%\draw[thick,->] (0,0,0) -- (0,0,1) node[anchor=south]{$z(k)$};

\draw[dashed, color=red] (O) -- (Px);
\draw[dashed, color=red] (O) -- (Py);
\draw[dashed, color=red] (O) -- (Pz);
\draw[dashed, color=red] (Px) -- (Pxy);
\draw[dashed, color=red] (Py) -- (Pxy);
\draw[dashed, color=red] (Px) -- (Pxz);
\draw[dashed, color=red] (Pz) -- (Pxz);
\draw[dashed, color=red] (Py) -- (Pyz);
\draw[dashed, color=red] (Pz) -- (Pyz);
\draw[dashed, color=red] (Pxy) -- (P);
\draw[dashed, color=red] (Pxz) -- (P);
\draw[dashed, color=red] (Pyz) -- (P);

\fill (0.0,0.0) circle (0.3mm) node[anchor=east]{1};
\fill (Px) circle (0.3mm) node[anchor=east]{2};
\fill (Pxy) circle (0.3mm) node[anchor=east]{3};
\fill (Py) circle (0.3mm) node[anchor=east]{4};

\fill (Pz) circle (0.3mm) node[anchor=east]{5};
\end{tikzpicture}


\end{figure}

%(i,j,k), (i+1,j,k), (i+1,j+1,k), (
%
%Pick an index i,j,k.
%    for (int k=0; k<nz; ++k)
%    {
%        for (int j=0; j<ny; ++j)
%        {
%            for (int i=0; i<nx; ++i)
%            {
%                cells.push_back(std::vector<IJK>{IJK(i,j,k), IJK(i+1, j, k), IJK(i+1, j+1, k), IJK(i, j+1, k), IJK(i,j,k+1), IJK(i+1, j, k+1), IJK(i+1, j+1, k+1), IJK(i, j+1, k+1)});
%            }
%        }
%    }


\end{document}
